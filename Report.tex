\documentclass{article}
\usepackage{lipsum}
\usepackage[nottoc,numbib]{tocbibind}
\usepackage[
backend=bibtex,
style=authoryear,
citestyle=authoryear
]{biblatex}
\usepackage{tikz-cd}
\usepackage{MathNotes}

%CONFIG
\renewcommand*\contentsname{Summary}

\lstdefinestyle{Hask}{
  language=Haskell,
  frame=single,
  tabsize=2,
  breaklines=false,
  captionpos=b
}

\addbibresource{Report.bib}

%\lstset{
%  literate={~} {$\sim$}{1}
%}


%TOPMATTER
\title{Category Theory's Role in Haskell}
\author{Jared Windover}
\date{\today}

%DOCUMENT

\begin{document}

\pagenumbering{roman}

\makeatletter
\begin{titlepage}
\begin{center}
{\Huge \@title}\\
{\Large Prepared for the University of the West Indies}\\[1em]
{\LARGE \@author}\\
{\large University of Waterloo\\
Ontario, Canada\\
\texttt{jaredwindover@gmail.com}}\\[0.8em]
{\Large \@date}
\end{center}
\begin{abstract}
\lipsum[1-2]
\end{abstract}
\end{titlepage}
\makeatother

\setcounter{page}{2}

\tableofcontents
\newpage

\pagenumbering{arabic}

\section{Introduction}

\begin{algex}[My algorithm]
  \begin{lstlisting}[style=Hask,caption={This is an algorithm for calculating the fibonacci numbers ~\parencite{LYHGG}.}]
    fib :: Integer n => n -> n
    fib 0 = 1
    fib n = (fib $ n - 1) + (fib $ n - 2)
  \end{lstlisting}
\end{algex} 
The following is the quadratic formula as shown in ~\parencite{RWH}:
\begin{ex}
  \begin{equation}
    x = \frac{-b \pm \sqrt{b^2 - 4ac}}{2a}
  \end{equation}
\end{ex}

\begin{ex}
  \begin{equation}
    \begin{tikzcd}
      A \arrow[yshift=.7ex,r,"f"]
      \arrow[yshift=-.7ex,r,swap,"g"]
      &
      B \arrow[r,"\ep"]
      \arrow[dr,swap,"h"]
      &
      C \arrow[densely dotted,d,"\not\exists"]
      \\
      & &
      D
    \end{tikzcd}
  \end{equation}
\end{ex}

Everything else is contained in ~\parencite{RWH}, ~\parencite{ASF}, and ~\parencite{TTT}.
\section{Category Theory Primer}
\section{Haskell Primer}
\section{Functors}
\subsection{Functors in Category Theory}
\subsection{Functors in Haskell}
\subsection{Comparison of Functors}
\section{Monoids}
\subsection{Monoids in Category Theory}
\subsection{Monoids in Haskell}
\subsection{Comparison of Monoids}
\section{Monads}
\subsection{Monads in Category Theory}
\subsection{Monads in Haskell}
\subsection{Comparison of Monads}

\newpage
\printbibliography
\end{document}